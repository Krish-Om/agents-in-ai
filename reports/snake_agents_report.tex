\documentclass[11pt,a4paper]{article}
\usepackage[utf8]{inputenc}
\usepackage[margin=1in]{geometry}
\usepackage{graphicx}
\usepackage{amsmath}
\usepackage{listings}
\usepackage{float}
\usepackage{subcaption}
\usepackage{xcolor}
\usepackage{booktabs}

% Configure code listings
\lstset{
    language=Python,
    basicstyle=\ttfamily\footnotesize,
    keywordstyle=\color{blue},
    commentstyle=\color{green!60!black},
    stringstyle=\color{red},
    breaklines=true,
    frame=single,
    numbers=left,
    numberstyle=\tiny\color{gray}
}

\title{Lab1: Intelligent Agents}

\begin{document}

\maketitle

\section{Introduction}

Intelligent agents represent fundamental building blocks of artificial intelligence systems, characterized by their ability to perceive environmental states and execute actions to achieve specific objectives. The Snake game provides an ideal testbed for comparing different agent architectures due to its well-defined state space, clear objectives, and dynamic challenges requiring both tactical and strategic decision-making.

This study implements and compares five distinct agent architectures: Simple Reflex Agent, Goal-Based Agent, Utility-Based Agent, Model-Based Agent, and Q-Learning Agent. Each represents a different paradigm in artificial intelligence, from reactive systems that respond to immediate stimuli to learning systems that adapt through experience.

\subsection{Objectives}

The primary objectives of this research are:
\begin{itemize}
\item Implement five different intelligent agent architectures for the Snake game environment
\item Compare performance characteristics and decision-making capabilities of each agent type
\item Analyze the progression from reactive to learning-based artificial intelligence
\item Evaluate the trade-offs between computational complexity and performance effectiveness
\item Demonstrate the evolution from hand-coded intelligence to machine-learned intelligence
\end{itemize}

\subsection{Methodology}

Each agent was implemented using Python and tested in a consistent Snake game environment. The evaluation methodology includes:
\begin{itemize}
\item Implementation of agent-specific decision-making algorithms
\item Performance measurement through gameplay sessions
\item Comparative analysis of behavioral patterns and strategies
\item Documentation of code implementations and architectural decisions
\end{itemize}

% ===================== AGENT IMPLEMENTATIONS =====================
\section{Agent Implementations}

This section presents the implementation details and experimental results for each intelligent agent architecture.

\subsection{Simple Reflex Agent}

\subsubsection{Description}
The Simple Reflex Agent implements the most basic form of intelligent behavior, making decisions based solely on the current perceptual input. This agent reacts to the immediate position of the apple relative to the snake's head, attempting to move toward the target if the path is safe. The agent operates without memory, planning capabilities, or learning mechanisms.

\subsubsection{Implementation}
\begin{lstlisting}[caption=Simple Reflex Agent]
def simple_agent(game):
    head_x, head_y = game.snake.x[0], game.snake.y[0]
    apple_x, apple_y = game.apple.x, game.apple.y

    if apple_x < head_x:
        next_x, next_y = game._get_potential_head("left")
        if not game._is_potential_move_colliding(next_x, next_y):
            game.snake.move_left()
            return
    if apple_x > head_x:
        next_x, next_y = game._get_potential_head("right")
        if not game._is_potential_move_colliding(next_x, next_y):
            game.snake.move_right()
            return
    if apple_y > head_y:
        next_x, next_y = game._get_potential_head("down")
        if not game._is_potential_move_colliding(next_x, next_y):
            game.snake.move_down()
            return
    if apple_y < head_y:
        next_x, next_y = game._get_potential_head("up")
        if not game._is_potential_move_colliding(next_x, next_y):
            game.snake.move_up()
            return
# No fallback strategy
\end{lstlisting}

\subsubsection{Results and Analysis}
The Simple Reflex Agent demonstrates basic navigation capabilities but exhibits significant limitations in complex scenarios. The agent successfully moves toward the apple when direct paths are available but fails when obstacles require strategic maneuvering. Performance is limited by the absence of planning and memory mechanisms.

\subsubsection{Screenshots}
\begin{figure}[H]
    \centering
    \begin{subfigure}{0.45\textwidth}
        \includegraphics[width=\textwidth]{ss/simple_play.png}
        \caption{Gameplay demonstration}
    \end{subfigure}
    \hfill
    \begin{subfigure}{0.45\textwidth}
        \includegraphics[width=\textwidth]{ss/simple_score.png}
        \caption{Performance metrics}
    \end{subfigure}
    \caption{Simple Reflex Agent implementation results}
\end{figure}

\subsection{Goal-Based Agent}

\subsubsection{Description}
The Goal-Based Agent represents a significant advancement over reactive systems by incorporating explicit goal formulation and planning mechanisms. This agent defines clear objectives including collision avoidance, apple acquisition, score maximization, and safety maintenance. The implementation utilizes sophisticated search algorithms including A* and Breadth-First Search (BFS) to compute optimal paths to target locations.

\subsubsection{Implementation}
\begin{lstlisting}[caption=Goal-Based Agent]
import math
from typing import List, Tuple, Optional
from collections import deque
import heapq
SIZE = 40
class Goals:
    REACH_APPLE = "reach_apple"
    AVOID_DEATH = "avoid_death"
    MAXIMIZE_SCORE = "maximize_score"
    MAINTAIN_SAFETY = "maintain_safety"
def goal_based_agent(game):
    """
    A true goal-based agent that:
    1. Defines explicit goals
    2. Plans sequences of actions using search algorithms
    3. Considers future states beyond immediate moves
    4. Uses A* pathfinding to achieve goals
    """
    # Get current state
    head_x, head_y = game.snake.x[0], game.snake.y[0]
    apple_x, apple_y = game.apple.x, game.apple.y

    # Goal Priority System
    current_goals = determine_active_goals(game)

    # Goal 1: PRIMARY - Find safe path to apple (REACH_APPLE + AVOID_DEATH)
    if Goals.REACH_APPLE in current_goals:
        path_to_apple = a_star_search(game, (head_x, head_y), (apple_x, apple_y))

        if path_to_apple and len(path_to_apple) > 1:
            # Verify the path is still safe after planning
            if is_path_safe(game, path_to_apple):
                next_move = get_direction_from_positions(
                    head_x, head_y, path_to_apple[1][0], path_to_apple[1][1]
                )
                execute_move(game, next_move)
                return

    # Goal 2: SAFETY - Maintain safe space when no direct path to apple
    if Goals.MAINTAIN_SAFETY in current_goals:
        safe_exploration_move = find_safe_exploration_move(game)
        if safe_exploration_move:
            execute_move(game, safe_exploration_move)
            return

    # Goal 3: MAXIMIZE_SCORE - Use BFS to find any reachable apple path
    if Goals.MAXIMIZE_SCORE in current_goals:
        bfs_path = bfs_search(game, (head_x, head_y), (apple_x, apple_y))
        if bfs_path and len(bfs_path) > 1:
            next_move = get_direction_from_positions(
                head_x, head_y, bfs_path[1][0], bfs_path[1][1]
            )
            execute_move(game, next_move)
            return

    # Goal 4: AVOID_DEATH - Emergency survival mode
    emergency_move(game)
\end{lstlisting}

\subsubsection{Results and Analysis}
The Goal-Based Agent demonstrates superior performance compared to the Simple Reflex Agent through strategic planning and multi-step lookahead capabilities. The hierarchical goal system enables intelligent priority management, with survival taking precedence over score optimization. The A* search algorithm provides optimal pathfinding when computational resources permit, while BFS serves as a fallback mechanism for broader exploration.

\subsubsection{Screenshots}
\begin{figure}[H]
    \centering
    \begin{subfigure}{0.45\textwidth}
        \includegraphics[width=\textwidth]{ss/goal_based_play.png}
        \caption{Strategic gameplay demonstration}
    \end{subfigure}
    \hfill
    \begin{subfigure}{0.45\textwidth}
        \includegraphics[width=\textwidth]{ss/goal_based_result.png}
        \caption{Performance analysis}
    \end{subfigure}
    \caption{Goal-Based Agent implementation results}
\end{figure}

% ===================== UTILITY-BASED AGENT =====================
\subsection{Utility-Based Agent}

\subsubsection{Description}
The Utility-Based Agent implements a sophisticated decision-making framework by evaluating each possible action through a comprehensive utility function. This agent considers multiple factors including food attraction, safety assessment, spatial availability, and movement efficiency. The utility-based approach allows for flexible preference modeling and multi-criteria optimization, representing a significant advancement over simple goal-based systems.

\subsubsection{Implementation}
\begin{lstlisting}[caption=Utility-Based Agent]
def utility_based_agent(game):
    """
    Utility-Based Agent that evaluates actions based on a utility function:
    1. Considers multiple factors: food attraction, safety, space, efficiency
    2. Computes a utility score for each possible actionw
    3. Selects the action with the highest utility score
    """
    head_x, head_y = game.snake.x[0], game.snake.y[0]
    apple_x, apple_y = game.apple.x, game.apple.y

    def compute_utility(action: str) -> float:
        """Compute utility score for a given action"""
        next_x, next_y = game._get_potential_head(action)

        # Basic safety check
        if game._is_potential_move_colliding(next_x, next_y):
            return float("-inf")  # Avoid dangerous moves

        utility = 0.0

        # Factor 1: Food attraction (distance to apple)
        distance_to_apple = abs(apple_x - next_x) + abs(apple_y - next_y)
        utility -= distance_to_apple  # Closer is better

        # Factor 2: Safety (avoid danger zones)
        if (next_x, next_y) in danger_zones:
            utility += 100  # Penalize moves into danger zones

        # Factor 3: Space (prefer moves with more free space)
        free_space = count_free_space_around(game, next_x, next_y)
        utility += free_space  # More free space is better

        # Factor 4: Efficiency (penalize unnecessary moves)
        if action in ["left", "right"]:
            utility -= 1  # Penalize horizontal moves
        elif action in ["up", "down"]:
            utility -= 1  # Penalize vertical moves

        return utility

    # Evaluate all possible actions and select the best one
    possible_actions = ["left", "right", "up", "down"]
    best_action = max(possible_actions, key=compute_utility)

    # Execute the best action
    if best_action == "left":
        game.snake.move_left()
    elif best_action == "right":
        game.snake.move_right()
    elif best_action == "up":
        game.snake.move_up()
    elif best_action == "down":
        game.snake.move_down()
\end{lstlisting}

\subsubsection{Results and Analysis}
The Utility-Based Agent demonstrates sophisticated decision-making capabilities through its multi-factor evaluation system. The utility function successfully balances competing objectives, showing improved performance in scenarios requiring trade-offs between immediate rewards and long-term safety. The agent exhibits flexible behavior adaptation based on weighted preference combinations.

\subsubsection{Screenshots}
\begin{figure}[H]
    \centering
    \begin{subfigure}{0.45\textwidth}
        \includegraphics[width=\textwidth]{ss/utility_based_play.png}
        \caption{Multi-criteria decision-making gameplay}
    \end{subfigure}
    \hfill
    \begin{subfigure}{0.45\textwidth}
        \includegraphics[width=\textwidth]{ss/utility_based_score.png}
        \caption{Performance optimization results}
    \end{subfigure}
    \caption{Utility-Based Agent implementation results}
\end{figure}

\subsection{Model-Based Agent}

\subsubsection{Description}
The Model-Based Agent represents a paradigm shift toward adaptive intelligence through the maintenance of an internal world model and experience-based learning mechanisms. This agent continuously updates its understanding of game dynamics, remembers dangerous positions, tracks successful strategies, and adapts its decision-making based on accumulated knowledge. The implementation demonstrates key principles of model-based reinforcement learning and adaptive behavior.

\subsubsection{Implementation}
\begin{lstlisting}[caption=Model-Based Agent]
def model_based_agent(game):
    """
    Model-Based Agent with persistent world model and learning capabilities
    
    Key Requirements Met:
    1. Internal State Storage
    2. World Model (game mechanics understanding)  
    3. Learning from Experience
    4. Model-based Decision Making
    """
    global _world_model

    # REQUIREMENT 1: Internal State Storage
    if _world_model is None:
        _world_model = {
            "apple_positions": [],      # Track apple movement pattern
            "danger_zones": set(),      # Remember dangerous positions
            "safe_moves": {},          # Remember successful moves
            "collision_near_misses": [], # Learn from close calls
            "movement_history": [],     # Track movement patterns
            "step_count": 0,           # Track game progression
        }

    world_model = _world_model
    world_model["step_count"] += 1

    # REQUIREMENT 2: World Model - Update understanding of game mechanics
    update_world_model(world_model, current_state, game)

    # REQUIREMENT 3: Learning from Experience
    learn_from_experience(world_model, current_state, game)

    # REQUIREMENT 4: Model-based Decision Making
    action = make_model_based_decision(world_model, current_state, game)
\end{lstlisting}

\subsubsection{Results and Analysis}
The Model-Based Agent exhibits sophisticated adaptive behavior through continuous learning and world model maintenance. The agent demonstrates improved performance over time by accumulating knowledge about dangerous positions and successful strategies. The persistent memory mechanism enables the agent to avoid previously encountered pitfalls and optimize decision-making based on historical experience.

\subsubsection{Screenshots}
\begin{figure}[H]
    \centering
    \begin{subfigure}{0.45\textwidth}
        \includegraphics[width=\textwidth]{ss/model_based_play.png}
        \caption{Adaptive gameplay demonstration}
    \end{subfigure}
    \hfill
    \begin{subfigure}{0.45\textwidth}
        \includegraphics[width=\textwidth]{ss/model_based_score.png}
        \caption{Learning performance metrics}
    \end{subfigure}
    \caption{Model-Based Agent implementation results}
\end{figure}

\subsection{Q-Learning Agent (Reinforcement Learning)}

\subsubsection{Description}
The Q-Learning Agent represents the pinnacle of autonomous intelligence in this study, implementing reinforcement learning principles to discover optimal strategies through trial-and-error exploration. This agent maintains a Q-table containing quality scores for state-action pairs, utilizing a simplified state representation to address the curse of dimensionality inherent in complex game environments. The learning mechanism incorporates reward signals (+50 for apple acquisition, -100 for collisions, -1 per movement) and employs the Bellman equation for value function updates.

\subsubsection{Implementation}
\begin{lstlisting}[caption=Q-Learning Agent]
def learning_model(game):
    """
    Q-Learning Agent that learns optimal strategies through experience:
    1. Uses simplified state representation (apple direction, distance, dangers)
    2. Maintains Q-table with quality scores for state-action pairs
    3. Employs epsilon-greedy exploration/exploitation strategy
    4. Learns from rewards and updates Q-values using Bellman equation
    5. Saves/loads knowledge for persistent learning across sessions
    """
    global last_state, last_action, last_score, game_count
    
    # Load Q-table on first run
    if game_count == 0:
        load_q_table()
        game_count += 1
    
    # Get simplified state representation
    current_state = get_simple_state(game)
    current_score = game.snake.length - 1
    
    # Get valid actions (avoid U-turns and collisions)
    current_direction = game.snake.direction
    all_actions = ["left", "right", "up", "down"]
    opposite = {"left": "right", "right": "left", "up": "down", "down": "up"}
    
    valid_actions = []
    for action in all_actions:
        if action != opposite.get(current_direction):
            next_x, next_y = game._get_potential_head(action)
            if not game._is_potential_move_colliding(next_x, next_y):
                valid_actions.append(action)
    
    # Update Q-table with previous experience
    if last_state is not None and last_action is not None:
        reward = get_reward(game, last_score, current_score, False)
        update_q_table(last_state, last_action, reward, current_state, valid_actions)
    
    # Choose action using epsilon-greedy strategy
    if valid_actions:
        action = choose_action(current_state, valid_actions)
        
        # Execute chosen action
        if action == "left":
            game.snake.move_left()
        elif action == "right":
            game.snake.move_right()
        elif action == "up":
            game.snake.move_up()
        elif action == "down":
            game.snake.move_down()
        
        # Store state for next iteration
        last_state = current_state
        last_action = action
        last_score = current_score

def get_simple_state(game):
    """Create simplified state representation to avoid curse of dimensionality"""
    head_x, head_y = game.snake.x[0], game.snake.y[0]
    apple_x, apple_y = game.apple.x, game.apple.y
    
    # Relative apple position
    horizontal = "right" if apple_x > head_x else "left" if apple_x < head_x else "same"
    vertical = "down" if apple_y > head_y else "up" if apple_y < head_y else "same"
    
    # Distance bucket
    distance = abs(apple_x - head_x) + abs(apple_y - head_y)
    dist_bucket = ("very_close" if distance <= 40 else "close" if distance <= 120 
                   else "medium" if distance <= 240 else "far")
    
    # Danger detection for each direction
    dangers = {}
    for direction in ["left", "right", "up", "down"]:
        next_x, next_y = game._get_potential_head(direction)
        dangers[direction] = game._is_potential_move_colliding(next_x, next_y)
    
    # Combine into state string
    state = f"{horizontal}_{vertical}_{dist_bucket}_{game.snake.direction}_"
    state += f"L{dangers['left']}_R{dangers['right']}_U{dangers['up']}_D{dangers['down']}"
    return state

def choose_action(state, valid_actions):
    """Epsilon-greedy action selection: 10% exploration, 90% exploitation"""
    if random.random() < 0.1:  # Explore
        return random.choice(valid_actions)
    else:  # Exploit
        best_action = max(valid_actions, key=lambda a: q_table[state][a])
        return best_action
\end{lstlisting}

\subsubsection{Key Features and Implementation Details}
The Q-Learning implementation incorporates several sophisticated mechanisms for effective learning:

\begin{itemize}
\item \textbf{State Space Reduction}: Utilizes simplified features including apple direction, distance classification, and collision dangers rather than full grid representation to mitigate the curse of dimensionality
\item \textbf{Q-Learning Algorithm}: Implements tabular Q-learning with Bellman equation updates for value function approximation
\item \textbf{Exploration vs Exploitation}: Employs epsilon-greedy strategy with 10\% random exploration and 90\% greedy exploitation for balanced learning
\item \textbf{Reward Engineering}: Structured reward system with +50 for apple acquisition, -100 for collisions, and -1 per movement to encourage efficient gameplay
\item \textbf{Persistent Learning}: Maintains Q-table persistence through JSON serialization for continuous improvement across gaming sessions
\item \textbf{Learning Parameters}: Configured with learning rate α=0.1 and discount factor γ=0.95 for optimal convergence
\end{itemize}

\subsubsection{Training Methodology and Metrics}
The Q-Learning agent incorporates comprehensive training measurement systems to evaluate learning progress and effectiveness:

\begin{table}[H]
\centering
\caption{Training Metrics and Measurements}
\begin{tabular}{@{}ll@{}}
\toprule
\textbf{Metric} & \textbf{Description} \\
\midrule
Episodes Completed & Number of complete games played (start to collision) \\
Total Steps/Actions & Cumulative moves made across all episodes \\
Best Score Achieved & Highest number of apples eaten in a single episode \\
Average Score Trend & Rolling average performance over last 100 episodes \\
Exploration Rate & Percentage of random actions vs learned actions \\
Q-Table Size & Number of unique state-action pairs learned \\
Episode Length & Average survival time showing learning progress \\
Training Time & Total time spent in learning sessions \\
\bottomrule
\end{tabular}
\end{table}

\textbf{Training Quality Indicators:}
\begin{itemize}
\item \textit{Good Training}: Rising average scores, expanding Q-table coverage, increasing episode duration
\item \textit{Learning Progress}: Decreasing exploration dependency, improving optimal score achievement
\item \textit{Convergence}: Performance stabilization following sufficient training episodes (100+ iterations)
\end{itemize}

\subsubsection{Results and Analysis}
The Q-Learning Agent demonstrates the remarkable capability of autonomous learning, progressively improving performance through experience accumulation. Initial episodes show random exploratory behavior, gradually transitioning to strategic gameplay as the Q-table develops. The agent exhibits superior long-term learning potential compared to all previous implementations, representing true adaptive artificial intelligence.

\subsubsection{Screenshots}
\begin{figure}[H]
    \centering
    \begin{subfigure}{0.45\textwidth}
        \includegraphics[width=\textwidth]{ss/learning_play.png}
        \caption{Reinforcement learning gameplay}
    \end{subfigure}
    \hfill
    \begin{subfigure}{0.45\textwidth}
        \includegraphics[width=\textwidth]{ss/learning_score.png}
        \caption{Performance evolution over training}
    \end{subfigure}
    \caption{Q-Learning Agent implementation results}
\end{figure}

% ===================== COMPARATIVE ANALYSIS =====================
\section{Comparative Analysis and Results}

This section presents a comprehensive comparative evaluation of the five implemented intelligent agent architectures, analyzing their performance characteristics, computational requirements, and behavioral patterns.
\begin{table}[H]
\centering
\caption{Agent Comparison}
\begin{tabular}{@{}lccccc@{}}
\toprule
\textbf{Characteristic} & \textbf{Simple Reflex} & \textbf{Goal-Based} & \textbf{Utility-Based} & \textbf{Model-Based} & \textbf{Q-Learning} \\
\midrule
Decision Making & Reactive & Proactive & Utility Optimization & Adaptive & Learned Policy \\
Planning Horizon & 1 step & Multi-step & Multi-factor & Learned Experience & Experience-Based \\
Search Algorithm & None & A*, BFS & Utility Function & Model-Based Eval & Q-Learning \\
Learning Capability & None & None & None & Experience-Based & Reinforcement \\
Memory & None & Temporary & None & Persistent & Q-Table \\
State Representation & Current Position & Full State & Current State & World Model & Simplified Features \\
Adaptation & None & None & None & Strategy Learning & Policy Learning \\
Performance & Moderate & Superior & Flexible & Optimal & Improving \\
\bottomrule
\end{tabular}
\end{table}

\section{Conclusion}

This research successfully implemented and evaluated five intelligent agent architectures for the Snake game environment, demonstrating the evolution from basic reactive systems to sophisticated learning-based artificial intelligence.

\subsection{Key Findings}
The experimental results demonstrate clear performance hierarchies among the implemented agent types:

\begin{itemize}
\item \textbf{Simple Reflex Agent}: Basic reactive behavior with limited planning capabilities
\item \textbf{Goal-Based Agent}: Strategic planning through A* and BFS search algorithms
\item \textbf{Utility-Based Agent}: Multi-criteria decision-making with flexible preference modeling
\item \textbf{Model-Based Agent}: Adaptive intelligence through experience accumulation and world model maintenance
\item \textbf{Q-Learning Agent}: Autonomous learning through reinforcement learning principles
\end{itemize}

The progression from hand-coded intelligence to machine-learned intelligence illustrates fundamental principles of artificial intelligence development. The Q-Learning agent represents the apex of autonomous learning, capable of discovering optimal strategies without prior knowledge and demonstrating true adaptive artificial intelligence.

\end{document}